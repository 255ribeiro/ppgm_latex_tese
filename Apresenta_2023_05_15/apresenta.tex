\documentclass[10pt]{beamer}

\usetheme{metropolis}
\usepackage{appendixnumberbeamer}

\usepackage{booktabs}
\usepackage[scale=2]{ccicons}

\usepackage{pgfplots}
\usepgfplotslibrary{dateplot}

\usepackage{xspace}
\newcommand{\themename}{\textbf{\textsc{metropolis}}\xspace}

\usepackage[brazilian,hyperpageref]{backref}	 % Paginas com as citações na bibl
\usepackage[alf]{abntex2cite}	% Citações padrão ABNT

%Tabelas
\usepackage{tabularx}
\usepackage{adjustbox}
\usepackage{pgfplotstable}

% Equações
\newcommand{\dmc}{\(DMC_x^2\) }
\newcommand{\pdcca}{\({\rho}_{DCCA}\) }
\newcommand{\fdfa}{\(F_{DFA}\) }

% Dados da apresentação

\title{\dmc e aprendizado de máquina aplicados à análise de dados se séries temporais do clima}
\subtitle{Apresentação do andamento da pesquisa}
%\date{\today}
\date{15/05/2023}
\author{Fernando Ferraz Ribeiro}
\institute{UEFS PPGM}

% \titlegraphic{\includegraphics[height=1.5cm]{../Figures/Logo_Uefs.jpg}}

\titlegraphic{\hfill\includegraphics[height=1.3cm]{../Figures/Logo_Uefs.jpg}\includegraphics[height=1.3cm]{../Figures/logoPPGM-UEFS.png}}

\begin{document}

\maketitle

\begin{frame}{Sumário}
  \setbeamertemplate{section in toc}[sections numbered]
  \tableofcontents[hideallsubsections]
\end{frame}

\section{Introdução}

\begin{frame}[fragile]{Premissas}

  

\end{frame}

\begin{frame}{Mapa Exemplo}


  
\end{frame}

\begin{frame}{Objetivo Principal}

  Investigar as correlações entre as variáveis climáticas através do coeficiente \dmc e utilizar o conhecimento destas correlações para alimentar um modelo preditivo do clima.

  
\end{frame}

\begin{frame}{Objetivos Gerais}

  \begin{enumerate}
    \label{enum:obj_espec}
    \item Implementar um algoritmo computacional geral para calcular o \dmc para qualquer número de séries temporais.
    \item Analisar um conjunto de dados climáticos contendo medições meteorológicas de todas as capitais brasileiras.
    \item Analisar um conjunto de dados meteorológicos sobre radiação solar com estações locadas em diversas partes do globo.
    \item Desenvolver e implementar um algoritmo de predição baseado em aprendizado de máquina e redes neurais artificias agregados com o coeficiente \dmc.
\end{enumerate}

\end{frame}


\section{Fundamentação Teórica}

\begin{frame}{Metodologia}

  Os dados que tem resolução do município e englobam uma parcela significativa dos municípios da Bahia, serão representados no arquivo dos municípios da Bahia. Conjuntos de dados que apresentam escala dos estados, serão representados no mapa dos estados do Brasil, Conjunto que apresentem apenas as capitais e cidades maiores, representaremos nos mapas dos municípios brasileiros.

\end{frame}

\begin{frame}{DFA - Algoritmo}

  \begin{enumerate}
    \label{dfa}
    \item Pegando a série temporal \(\{x_{i}\}\) com  \(i\) variando de  \(1\) à \(N\), a série integrada \(X_{k}\) é calculada por \(X_{k} = \sum_{i=1}^{k}\left[x_{i} - \langle x \rangle \right] \) com \(k\) também variando entre \(1\) e \(N\);
    \item A série  \(X_{k}\) e dividida em \(N - n\) caixas de tamanha\(n\) (escala temporal), cada caixa contendo \(n + 1\) observações, iniciando em \(i\) até \(i + n\);
    \item Para cada caixa um polinômio (geralmente de grau 1) é ajustado, gerando \(\widetilde{X}_{k, i}\) with \( i \le k \le (i + n) \) eliminando assim a tendência (detrended values);
    \item  para cada caixa é calculado: \(f_{DFA}^{2}(n, i) = \frac{1}{1+n} \sum_{k=i}^{i + n}(X_{k}-\widetilde{X}_{k, i})^{2}\)
    \item Para todas as caixas de umaescala temporal o DFA é calculado como: \(F_{DFA}(n) = \sqrt{\frac{1}{N - n} \sum_{i=1}^{N-n} f_{DFA}^{2}(n, i)}\);
    \item Para um número de diferentes escalas temorais (n), com valores possíveis entre \( 4 \le n \le \frac{N}{4}\), a função \(F_{DFA}\) é calculada para encontrar a relação entre \(F_{DFA} \times n\)
  \end{enumerate}
 

\end{frame}

\begin{frame}{Dados Google}




 


\end{frame}


\section{Resultados}

\begin{frame}{Resultados Esperados}

  Para cada um dos conjunto de dados que apresentem escala de município e incluam um grande conjunto de cidades do município da Bahia, serão representados em conjuntos de 7 mapas, representando em formato coroplético a mediana da mobilidade de cada município. A necessidade de normalização/padronização dos dados será avaliada caso a caso. 

  Os mapas que apresentem poucos cidades no território nacional, ou que apresente dados no nível d bairro será utilizada visualização por tamanho de ponto.
  
\end{frame}

\section{Considerações Finais}

\begin{frame}{Considerações Finais}

  As mudanças provocadas pela pandemia de COVID 19 em diversas áreas ainda precisa ser melhor entendido. No que diz respeito aos dados oriundos de dispositivos móveis de comunicação muito temos que analisar.

  Tanto das possibilidades e limitações, quanto as questões éticas e respeito à privacidade devem ser analisadas. A visualização de dados proposta neste estudo pretende fornecer subsídios para este debate. 


\end{frame}

\begin{frame}{Trabahos Futuros}

  Armazenamento das informações coletadas em um banco de dados georeferenciados.
  
\end{frame}

\section{Referências}

\begin{frame}[allowframebreaks]

  \bibliography{../References/referencias.bib}
  
\end{frame}

\end{document}
