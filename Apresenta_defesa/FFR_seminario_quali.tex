\documentclass[11pt, aspectratio=169]{beamer}

\usetheme{metropolis}
\usepackage{appendixnumberbeamer}

\usepackage{booktabs}
\usepackage[scale=2]{ccicons}

\usepackage{pgfplots}
\usepgfplotslibrary{dateplot}

\usepackage{xspace}
\newcommand{\themename}{\textbf{\textsc{metropolis}}\xspace}

\usepackage[brazilian,hyperpageref]{backref}	 % Paginas com as citações na bibl
\usepackage[alf]{abntex2cite}	% Citações padrão ABNT

%Tabelas
\usepackage{tabularx}
\usepackage{adjustbox}
\usepackage{pgfplotstable}

% Equações
\newcommand{\dmc}{\(DMC_x^2\) }
\newcommand{\pdcca}{\({\rho}_{DCCA}\) }
\newcommand{\fdfa}{\(F_{DFA}\) }

% Dados da apresentação

\title{MODELO DE APRENDIZADO DE MÁQUINA \\BASEADO NO \pdcca e \dmc}
\subtitle{Seminário de Qualificação}
%\date{\today}
\date{08/11/2024}
\author{Discente: Fernando Ferraz Ribeiro \\Orientador: Prof. Dr.Gilney Figueira Zebende \\Coorientador: Prof. Dr. Juan Alberto Leyva Cruz}



\institute{PPGM UEFS - Feira de Santana, BA}

\titlegraphic{\hfill\includegraphics[height=1.0cm]{../Figures/Logo_Uefs.jpg}\includegraphics[height=1.0cm]{../Figures/logoPPGM-UEFS.png}}

\begin{document}

\maketitle

\begin{frame}{Sumário}
  \setbeamertemplate{section in toc}[sections numbered]
  \tableofcontents[hideallsubsections]
\end{frame}

\section{Introdução}

\begin{frame}[fragile]{Sistemas Complexos}

  Este conjunto amplo de fenômenos é comumente identificado e agrupado por algumas de suas características: são formados pela contribuição de um conjunto (geralmente grande) de componentes (muitas vezes simples) que, interagindo, estruturam-se de forma auto-organizada, gerando resultados inesperados, que não podem ser previstos pelos estudos estatísticos e/ou matemáticos tradicionais dos elementos formadores do sistema.

\end{frame}

\begin{frame}
  \frametitle{Reconhecimento}

  Em 2021, a Academia Real das Ciências da Suécia concedeu metade do Prêmio Nobel de Física para Syukuro Manabe e Klaus Hasselmann, cujos estudos apresentam modelos complexos para a análise do clima. Em particular apontam uma correlação entre as emissões de dióxido de carbono e as mudanças climáticas.

\end{frame}

\begin{frame}
  \frametitle{Sistemas Complexos e Ciência de Dados}

  Muitos fenômenos complexos são investigados pela análise de grandes conjuntos de dados. É notável a velocidade e quantidade de dados que são gerados e armazenados pela humanidade atualmente. A aquisição, manipulação, gestão, armazenamento e criação de valor a partir de dados, através de ambientes computacionais, tem-se apresentado como um novo paradigma tecnológico. ...

\end{frame}

\begin{frame}
  \frametitle{Sistemas Complexos e Ciência de Dados}
  Um campo do conhecimento que recebeu a denominação de \textbf{Ciência de Dados}, conceito que envelopa alguns termos frequentemente associados à inovação científica, técnica e social como \textbf{Big Data, mineração de dados, Business Intelligence internet das coisas, inteligência artificial} e \textbf{aprendizado de máquina(AM)}, dentre outros \cite[p. 12-13]{EMCdata2015}.
\end{frame}

\begin{frame}
  \frametitle{Séries Temporais}

  As séries temporais são definidas como um conjunto de observações (numéricas ou categóricas) ordenado no tempo.  Embora muitos dos dados que descrevem as dinâmicas espaciais podem ser registrados na forma de sérias temporais (abastecimento de água nas tubulações, consumo de energia elétrica nos imóveis, fluxos de pessoas e veículos pela cidade, casos de uma doença por dia, etc.), contudo as técnicas de medição de correlações, bem como a devida exploração destas para inferir novos conhecimentos, permanecem como perguntas abertas em muitas sub-áreas das ciências ambientais\cite{Bermudez-Edo2018}.

\end{frame}

\begin{frame}{Proposta}

  Esta pesquisa propõe a exploração das potencialidades dos coeficientes \pdcca e \dmc como componentes de \textbf{Modelos de aprendizado de máquina(AM)}

\end{frame}

\begin{frame}{Objetivo Principal}

  O objetivo principal desta pesquisa é: investigar as potencialidades dos coeficientes \pdcca e \dmc como componentes na geração de \textbf{um modelo de aprendizado de máquina (AM)}.

\end{frame}

\begin{frame}{Objetivos Gerais}

  \begin{enumerate}
    \label{enum:obj_espec}
    \item Implementar um algoritmo computacional geral para calcular o \dmc para qualquer número de séries temporais.
    \item desenvolver um pacote Python \textbf{modular} que permita \textbf{flexibilidade} e interoperabilidade com o ecossistema de ferramentas de \textbf{AM} disponíveis para a linguagem;
    \item desenvolver e implementar um algoritmo de predição baseado em aprendizado de máquina e redes neurais artificias agregados com o coeficiente \dmc.
  \end{enumerate}

\end{frame}

\begin{frame}
  \frametitle{Premissas}

  \begin{enumerate}
    \label{enum:premissas}
    \item As ferramentas de \textbf{AM} são extremamente competentes na resolução de problemas complexos, baseados em grandes conjuntos de dados;
    \item o \dmc, pelas características de análise do método, pode ajudar a entender estas correlações;
    \item o \dmc~é uma generalização do método \pdcca~para múltiplas séries temporais;
    \item o \pdcca, em determinadas condições testadas, apresentou resultados mais interessantes (como melhor descrição dos fenômenos) que os apresentados pelo coeficiente de Pearson quando aplicado à séries temporais~\cite{Wang2013}.
  \end{enumerate}

\end{frame}

\begin{frame}
  \frametitle{Hipótese}

  \begin{enumerate}

    \item É possível criar uma modelo preditivo de \textbf{AM} que se beneficie dos coeficientes \pdcca e \dmc.
  \end{enumerate}

\end{frame}


\section{Metodologia}

\begin{frame}
  \frametitle{Aprendizado de Maquina - Machine Learning}
  %%%%%%%%%%%%%%%%%%%%%%%%
  \begin{figure}[ht]
    %%%%%%%
    \begin{minipage}[b]{0.45\linewidth}
      \centering
      \caption{Conceituação - AM}
      \includegraphics[height=.6\paperheight]{../Figures/ML/mat_est_ML.png}
      \\{\footnotesize Fonte: Elaborada pelos autores}
      \label{fig:diag_ml_01}
    \end{minipage}
    %%%%%%%
    \hspace{0.2cm}
    %%%%%%%
    \begin{minipage}[b]{0.45\linewidth}
      \centering
      \caption{Diagrama conceitual - AM}
      \includegraphics[height=.6\paperheight]{../Figures/ML/MAPA_conceitual_ciencia_de_dados_recorte.jpg}
      \\{\footnotesize Fonte: Elaborada pelos autores}
      \label{fig:diag_ml}
    \end{minipage}
    %%%%%%%
  \end{figure}
  %%%%%%%%%%%%%%%%%%%%%%%%
\end{frame}



\begin{frame}
  \frametitle{Aprendizado de Maquina - caracterização}


  \begin{Large}
    \begin{center}
      \begin{itemize}
        \item P (performance)- Desempenho do algoritmo...
        \item T (task) - na execução de uma tarefa...
        \item E (experience) - através da experiência (dados).
      \end{itemize}
    \end{center}
  \end{Large}
  \cite{mitchell1997}\\
  \url{https://www.cs.cmu.edu/~ninamf/courses/601sp15/index.html}

\end{frame}


\begin{frame}
  \frametitle{Princípios norteadores}

  \begin{Large}
    \begin{center}
      \begin{itemize}
        \item Conheça os dados
        \item Entenda os algoritmos
        \item Desconfie dos resultados
      \end{itemize}
    \end{center}
  \end{Large}

\end{frame}

\begin{frame}

  \frametitle{Etapas}

  \begin{enumerate}
    \item Implementação do algoritmo do \dmc;
    \item desenvolvimento da biblioteca;
    \item elaboração da arquitetura do modelo;
    \item Seleção de atributos;
    \item Validação e ajustes do modelo;
    \item Visualização e análise dos resultados.
  \end{enumerate}
\end{frame}


\begin{frame}
  \frametitle{Metodologia - Fluxograma}

  \begin{figure}[!htb]
    \centering
    \caption{Fluxograma metodológico}
    \includegraphics[height=.6\paperheight]{../Figures/intro/metodo.png}
    \\{\footnotesize Fonte: Elaborada pelos autores}
    \label{fig:metodo}
  \end{figure}


\end{frame}



\begin{frame}
  \frametitle{Análise exploratória}

  A análise exploratória dos dados tem por objetivo entender características, potenciais, limitações e possíveis erros na coleta dos dados de cada um dos conjuntos de dados.


\end{frame}


\begin{frame}
  \frametitle{Seleção de atributos}

  Em ciência de dados, a seleção de atributos é a escolha de um subconjunto de atributos ou variáveis são selecionados para uma determinada análise ou para a criação de um modelo.

  Nesta pesquisa utilizaremos alguns algoritmos de seleção de atributos para comparar os resultados com as correlações do \dmc, a saber:

  \begin{itemize}
    \item Time Series Feature Importance
    \item Mutual Information
    \item Autoencoder
    \item Random Forest Importance(?)
  \end{itemize}


\end{frame}



\begin{frame}
  \frametitle{Aplicação do \dmc}

  Selecionar variáveis de acordo com os algoritmos de seleção de atributos.

  Utilizar estratégias de visualização de dados e comparar os resultados do \dmc com as seleções de variáveis


\end{frame}

\begin{frame}
  \frametitle{Seleção de variáveis para aplicação do modelo preditivo}

  Baseado nas análises anteriores, escolher um conjunto de variáveis para iniciar a implementação do modelo.

\end{frame}


\begin{frame}
  \frametitle{Validação do Algoritmo de AM}

  \begin{figure}[!htb]
    \centering
    \caption{Diagrama de Grimm e Railsback}
    \includegraphics[height=.5\paperheight]{../Figures/intro/Ciclo_Grimm.png}
    \\{\footnotesize Fonte: Elaborada pelos autores}
    \label{fig:fluxoGrimm}
  \end{figure}



\end{frame}



\section{Fundamentação Teórica}


\begin{frame}{DFA - \cite{Peng_1994}}


  \begin{enumerate}
    \label{list:dfa}
    \item Pegando a série temporal \(\{x_{i}\}\) com  \(i\) variando de  \(1\) à \(N\), a série integrada \(X_{k}\) é calculada por \(X_{k} = \sum_{i=1}^{k}\left[x_{i} - \langle x \rangle \right] \) com \(k\) também variando entre \(1\) e \(N\);
    \item A série  \(X_{k}\) é dividida em \(N - n\) caixas de tamanho \(n\) (escala temporal), cada caixa contendo \(n + 1\) observações, iniciando em \(i\) até \(i + n\);
    \item Para cada caixa um polinômio (geralmente de grau 1) é ajustado, gerando \(\widetilde{X}_{k, i}\) com \( i \le k \le (i + n) \) eliminando assim a tendência (detrended values);
    \item  para cada caixa é calculado: \(f_{DFA}^{2}(n, i) = \frac{1}{1+n} \sum_{k=i}^{i + n}(X_{k}-\widetilde{X}_{k, i})^{2}\)
    \item Para todas as caixas de uma escala temporal o DFA é calculado como: \(F_{DFA}(n) = \sqrt{\frac{1}{N - n} \sum_{i=1}^{N-n} f_{DFA}^{2}(n, i)}\);
    \item Para um número de diferentes escalas temporais (n), com valores possíveis entre \( 4 \le n \le \frac{N}{4}\), a função \(F_{DFA}\) é calculada para encontrar a relação entre \(F_{DFA} \times n\)
  \end{enumerate}


\end{frame}

\begin{frame}
  \frametitle{DCCA - \cite{Podobnik2008}}
  \begin{enumerate}
    \label{list:dcca}
    \item Para duas séries temporais \(\{x_{\alpha, i}\}\) e \(\{x_{\beta, i}\}\) com  \(i\) variando de  \(1\) à \(N\), as séries integradas \(X_{\alpha, k}\) e \(X_{\beta, k}\) são calculadas por \(X_{k} = \sum_{i=1}^{k}\left[x_{i} - \langle x \rangle \right] \) com \(k\) também variando entre \(1\) e \(N\);
    \item As séries  \(X_{\alpha, k}\) e \(X_{\beta, k}\) são divididas em \(N - n\) caixas de tamanho \(n\) (escala temporal), cada caixa contendo \(n + 1\) observações, iniciando em \(i\) até \(i + n\);
    \item Para cada caixa um polinômio é ajustado, gerando \(\widetilde{X}_{k, i}\) para a primeira série e \(\widetilde{Y}_{k, i}\) para a segunda com \( i \le k \le (i + n) \) eliminando assim a tendência ;
    \item  Para cada caixa é calculado: $f_{DCCA}^{2}(n, i) = \frac{1}{1+n} \sum_{k=i}^{i + n}(X_{k}-\widetilde{X}_{k, i})(Y_{k}-\widetilde{Y}_{k, i})$
    \item Para todas as caixas de uma escala temporal a função $F_{DCCA}^{2}(n)$ é calculada por: $F_{DCCA}^{2}(n) = \frac{1}{N-n} \sum_{i=1}^{N-n} f_{DCCA}^{2}(n, i)$;
    \item Para um número de diferentes escalas temporais $(n)$, com valores possíveis entre \( 4 \le n \le \frac{N}{4}\), a função $F_{DCCA}^{2}(n)$ é calculada para encontrar a relação entre $F_{DCCA}^{2}(n) \times n$.
  \end{enumerate}

\end{frame}

\begin{frame}
  \frametitle{\pdcca - \cite{Zebende2011}}

  \begin{equation}
    \label{eq_pdcca}
    \rho_{DCCA}(n) = \frac{F_{DCCA}^2 (n)}{ F_{DFA1} (n) F_{DFA2} (n)}
  \end{equation}

\end{frame}

\begin{frame}
  \frametitle{\dmc - \cite{Zebende2018}}
  \begin{equation}\label{eq:dmc}
    {DMC}_{x}^{2}  \equiv \rho_{y,x_{i}}(n)^{T} \rho^{-1}(n) \rho_{y,x_{i}}(n)
  \end{equation}


  \begin{equation}\label{eq:dmc_mat_inv}
    \rho^{-1}(n) = \left(\begin{matrix}
      1                     & \rho_{x_{1},x_{2}}(n) & \rho_{x_{1},x_{3}}(n) & \dots & \rho_{x_{1},x_{i}}(n) \\
      \rho_{x_{2},x_{1}}(n) & 1                     & \rho_{x_{2},x_{3}}(n) & \dots & \rho_{x_{2},x_{i}}(n) \\
      \vdots                & \vdots                & \vdots                & \dots & \vdots                \\
      \rho_{x_{i},x_{1}}(n) & \rho_{x_{i},x_{2}}(n) & \rho_{x_{i},x_{3}}(n) & \dots & 1                     \\
    \end{matrix}\right)^{-1}
  \end{equation}

  \begin{equation}
    \rho_{Y,X_i}(n)^T=[\rho_{Y,X_1}(n), \rho_{Y,X_2}(n),\cdots,\rho_{Y,X_j}(n)]
  \end{equation}


\end{frame}



\section{Resultados}

\begin{frame}{Produtos da Tese}
  \begin{itemize}
    \item {Artigo: revisão de literatura.}
    \item Aplicativo de Cálculo do \dmc.
    \item Pacote Python para cálculo do DFA, DCCA, \pdcca, e  \dmc.
    \item Artigo: revista \emph{Journal of Statistical Software}.
    \item Implementação do modelo de AM usando \dmc.
    \item Artigo: validação do modelo AM.
    \item Artigo de aplicação.
  \end{itemize}


\end{frame}

\subsection{Artigo 01}

\begin{frame}
  \frametitle{Artigo 01 - Publicado}

  \begin{figure}[!htb]
    \centering
    \caption{\cite{Oliveira2023}}
    \includegraphics[height=.6\paperheight]{../Figures/artigos_publicados/artigo_01_abr_2023.png}
    \label{fig:ar_pub_01}
  \end{figure}

\end{frame}

% \section{Considerações Finais}

% \begin{frame}{Considerações Finais}
\begin{frame}
  \frametitle{Algoritmo registrado}

  \begin{figure}[!htb]
    \centering
    \caption{Registro de Software}
    \includegraphics[height=.6\paperheight]{../Figures/artigos_publicados/certificado_alg_01.png}
    \label{fig:alg_pub_01}
  \end{figure}

\end{frame}


\subsection{Algoritmo registrado - 01}

\begin{frame}
  \frametitle{Algoritmo registrado - 01}

  \begin{equation}
    \begin{split}
      DMC_{x}^{2} \quad = \quad & \Big( \Big. \rho^{2}_{X_{2},X_{3}} \times \rho^{2}_{Y,X_{1}}- \rho^{2}_{Y,X_{1}} + \rho^{2}_{X_{1},X_{3}}\times \rho^{2}_{Y,X_{2}}-\rho^{2}_{Y,X_{2}} \\
      &+ 2 \times \rho_{X_{1},X_{2}} \times \rho_{Y,X_{1}} \times \rho_{Y,X_{2}}   - 2 \times \rho_{X_{1},X_{3}} \times \rho_{X_{2},X_{3}} \times \rho_{Y,X_{1}} \\
      &+ \rho^{2}_{X_{1},X_{2}} \times \rho^{2}_{Y,X_{3}}-\rho^{2}_{Y,X_{3}} + 2 \times \rho_{X_{1},X_{3}} \times \rho_{Y,X_{1}} \times \rho_{Y,X_{3}} \\
      &- 2 \times \rho_{X_{1},X_{2}} \times \rho_{X_{2},X_{3}} \times \rho_{Y,X_{1}} \times \rho_{Y,X_{3}} \\
      &- 2 \times \rho_{X_{1},X_{2}} \times \rho_{X_{1},X_{3}} \times \rho_{Y,X_{2}} \times \rho_{Y,X_{3}} \\
      &+ 2 \times \rho_{X_{2},X_{3}} \times \rho_{Y,X_{2}} \times \rho_{Y,X_{3}} \Big. \Big)    \quad \Big/ \\
      & \Big( \Big. \rho^{2}_{X_{1},X_{2}} + \rho^{2}_{X_{1},X_{3}} + \rho^{2}_{X_{2},X_{3}} - 2 \times \rho_{X_{1},X_{2}} \times \rho_{X_{1},X_{3}} \times \rho_{X_{2},X_{3}}^{-1}\Big. \Big)  \\
    \end{split}
    \label{eq:dmc_3x_y}
  \end{equation}

\end{frame}

\subsection{Artigo 02}

\begin{frame}
  \frametitle{Artigo 02}

  \textbf{\Large{Multi Cross-correlation Analysis in a Multi-channel EEG applied in Motor Activity (Real/Imaginary)}}
  \medskip
  \begin{center}
    \url{https://255ribeiro.github.io/Multi_Cross-correlation_EEG/}
  \end{center}
 

\end{frame}

\begin{frame}
  \frametitle{Artigo 02}
  \begin{figure}[!h]
    \includegraphics[height=.6\paperheight]{../Figures/art_02/Fig1.png}
    \caption{Posição dos canais de EEG e canais utilizados}
    \label{fig01}
  \end{figure}
\end{frame}



\begin{frame}
  \frametitle{Artigo 02}

  \begin{figure}[!h]
    \includegraphics[height=.5\paperheight]{../Figures/art_02/Fig2.png}
    \caption{$DMC_{x}^{2}$ as a function of time scale $n$. Here is showing the results for subject S014 recordings for Task 2, presenting experiments 04, 08, 12 and the mean values for these experiments. The vertical line represents $n=67$ and $QoD$ is the total amount of time scales involved in $DMC_{x}^{2}$ calculations.}
    \label{fig02}
  \end{figure}
\end{frame}


\begin{frame}
  \frametitle{Artigo 02}

  %%%%%%%%%%%%%%%%%%%
  \begin{figure}[!h]
    \includegraphics[height=.5\paperheight]{../Figures/art_02/Fig3.jpg}
    \caption{Mean values of $DMC_{x}^{2} \times n$ for all Tasks: Left/Right (Imaginary), Left/Right (Real), Top/Down (Imaginary), and Top/Down (Real) for the S014 subject.}
    \label{fig03}
  \end{figure}
  %%%%%%%%%%%%%%%%%%%
\end{frame}


\begin{frame}
  \frametitle{Artigo 02}

  %%%%%%%%%%%%%%%%%%%
  \begin{figure}[!h]
    \includegraphics[height=.5\paperheight]{../Figures/art_02/Fig4.jpg}
    \caption{Mean values of $DMC_{x}^{2} \times n$ for all Tasks: Left/Right (Imaginary), Left/Right (Real), Top/Down (Imaginary), and Top/Down (Real) for the S036 subject.}
    \label{fig04}
  \end{figure}
  %%%%%%%%%%%%%%%%%%%
\end{frame}

\begin{frame}
  \frametitle{Artigo 02}

  %%%%%%%%%%%%%%%%%%%
  \begin{figure}[!h]
    \includegraphics[height=.5\paperheight]{../Figures/art_02/Fig5.jpg}
    \caption{Mean values of $DMC_{x}^{2} \times n$ for all Tasks: Left/Right (Imaginary), Left/Right (Real), Top/Down (Imaginary), and Top/Down (Real) for the S039 subject.}
    \label{fig05}
  \end{figure}
  %%%%%%%%%%%%%%%%%%%
\end{frame}


\begin{frame}
  \frametitle{Artigo 02}

  %%%%%s%art_02/%%%%%%%%%%%%%
  \begin{figure}[!h]
    \includegraphics[height=.5\paperheight]{../Figures/art_02/Fig6.jpg}
    \caption{Mean values of $DMC_{x}^{2} \times n$ for all Tasks: Left/Right (Imaginary), Left/Right (Real), Top/Down (Imaginary), and Top/Down (Real) for the S078 subject.}
    \label{fig06}
  \end{figure}
  %%%%%%%%%%%%%%%%%%%
\end{frame}

\begin{frame}
  \frametitle{Artigo 02}

  %%%%%%%%%%%%%%%%%%%
  \begin{figure}[!h]
    \includegraphics[height=.5\paperheight]{../Figures/art_02/Fig7.jpg}
    \caption{Mean values of $DMC_{x}^{2} \times n$ for all Tasks: Left/Right (Imaginary), Left/Right (Real), Top/Down (Imaginary), and Top/Down (Real) for the S099 subject.}
    \label{fig07}
  \end{figure}
  %%%%%%%%%%%%%%%%%%%
\end{frame}

\begin{frame}
  \frametitle{Artigo 02}

  %%%%%%%%%%%%%%%%%%%
  \begin{figure}[!h]
    \includegraphics[height=.5\paperheight]{../Figures/art_02/Fig8.jpg}
    \caption{$DMC_{x}^{2} \times n$ mean global for all Subjects and Tasks: Left/Right (Imaginary), Left/Right (Real), Top/Down (Imaginary), and Top/Down (Real).}
    \label{fig08}
  \end{figure}
  %%%%%%%%%%%%%%%%%%%
\end{frame}


\begin{frame}
  \frametitle{Artigo 02}

  %%%%%%%%%%%%%%%%%%%
  \begin{figure}[!h]
    \includegraphics[height=.5\paperheight]{../Figures/art_02/Fig9.jpg}
    \caption{Standard deviation, $sd$, of the global mean for all Subjects and Tasks.}
    \label{fig09}
  \end{figure}
  %%%%%%%%%%%%%%%%%%%
\end{frame}

\begin{frame}
  \frametitle{Artigo 02}


  %%%%%%%%%%%%%%%%%%%%%%%%
  \begin{figure}[ht]
    %%%%%%%
    \begin{minipage}[b]{0.21\linewidth}
      \centering
      \includegraphics[height=.5\paperheight]{../Figures/art_02/Fig10.jpg}
      \caption{MSE for the Channel $F_{3}$.}
      \label{fig21}

    \end{minipage}
    %%%%%%%
    \hspace{0.2cm}
    %%%%%%%
    \begin{minipage}[b]{0.21\linewidth}
      \centering
      \includegraphics[height=.5\paperheight]{../Figures/art_02/Fig11.jpg}
      \caption{MSE for the Channel $F_{6}$.}
      \label{fig22}
    \end{minipage}
    %%%%%%%
    \hspace{0.2cm}
    %%%%%%%
    \begin{minipage}[b]{0.21\linewidth}
      \centering
      \includegraphics[height=.5\paperheight]{../Figures/art_02/Fig12.jpg}
      \caption{MSE for the Channel $P_{3}$.}
      \label{fig23}
    \end{minipage}
    %%%%%%%
    \hspace{0.2cm}
    %%%%%%%
    \begin{minipage}[b]{0.21\linewidth}
      \centering
      \includegraphics[height=.5\paperheight]{../Figures/art_02/Fig13.jpg}
      \caption{MSE for the Channel $P_{6}$.}
      \label{fig24}
    \end{minipage}
    %%%%%%%

  \end{figure}
  %%%%%%%%%%%%%%%%%%%%%%%%
\end{frame}


\subsection{Algoritmo - biblioteca Python/Zig}

\begin{frame}
  \frametitle{Algoritmo - biblioteca Python/Zig}
  \begin{center}
    pip install zebende \\
    \url{https://pypi.org/project/zebende/}
  \end{center}

\end{frame}

\begin{frame}
  \frametitle{Algoritmo - biblioteca Python/Zig}

  %%%%%%%%%%%%%%%%%%%
  \begin{figure}[!h]
    \includegraphics[width=.8\paperwidth]{../Figures/pylib/pdcca_chart.png}
    \caption{\pdcca flowchart.}
    \label{chart_01}
  \end{figure}
  %%%%%%%%%%%%%%%%%%%


\end{frame}

\begin{frame}
  \frametitle{Algoritmo - biblioteca Python/Zig}

  %%%%%%%%%%%%%%%%%%%
  \begin{figure}[!h]
    \includegraphics[width=.8\paperwidth]{../Figures/pylib/dmc_chart.png}
    \caption{\dmc flowchart.}
    \label{chart_02}
  \end{figure}
  %%%%%%%%%%%%%%%%%%%


\end{frame}

\section{Aprendizado de máquina}


\begin{frame}
  \frametitle{Aprendizado de máquina}

  \begin{center}\frametitle{Seleção de variáveis}
    \large{Seleção de variáveis}
  \end{center}

  A seleção de características é o processo de reduzir o número de variáveis de entrada em um conjunto de dados para melhorar o desempenho e a eficiência de um modelo de aprendizado de máquina. É essencial selecionar as características mais relevantes, pois características excessivas ou irrelevantes podem levar ao \emph{overfitting}, à diminuição da precisão do modelo e ao aumento dos custos computacionais.
\end{frame}

\begin{frame}
  \begin{center}
    \large{Inspiração - ganho de informação}

    \begin{equation} H(Y) = - \sum_{i=1}^k p_i \log_2 p_i \end{equation}

    \begin{equation} IG(A, Y) = H(Y) - \sum_{v \in values(A)} \frac{|N_v|}{|N|} H(Y|A=v)
    \end{equation}
  \end{center}
\end{frame}

\begin{frame}
  \frametitle{Proposta}
  \begin{center}
    \begin{equation}
      \max_{(k = 1, n)} = {(\rho_{DCCA}(Y, x_k))}^2 - DMC_x^2(x_k: X_{i\neq k})
    \end{equation}

    \begin{equation}
      \max_{(k = 1, n-1)} = DMC_x^2(Y: x_{k1},x_{k2}) - DMC_x^2(x_{k1} : X_{i\neq \{k1, k2\}})
    \end{equation}
  \end{center}
\end{frame}

\section{Referências}

\begin{frame}[allowframebreaks]

  \bibliography{../References/referencias.bib}

\end{frame}

\section*{Obrigado}

\end{document}
