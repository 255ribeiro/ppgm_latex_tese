\chapter{Introdução}
\label{cap:introducao}


\begin{flushright}
``Ordinary life is pretty complex stuff.''\\
(Harvey Pekar)
\end{flushright}

Os sistemas complexos compreendem um campo interdisciplinar da ciência que não possui uma definição exata. Este campo procura estudar numericamente um conjunto amplo de fenômenos não determinísticos, formados pela contribuição de um conjunto (geralmente grande) de componentes (muitas vezes simples) que, interagindo, estruturam-se de forma auto-organizada, gerando resultados inesperados, que não podem ser previstos pelos estudos estatísticos e/ou matemáticos tradicionais dos elementos formadores do sistema.

Na área dos estudos ambientais, os sistemas complexos possuem diversas aplicações: sistemas de transportes, redes de energia e comunicação, organizações sociais e econômicas, densidade e ocupação humana do espaço, dentre outas. Os estudos do clima ocupam um espaço de particular relevância na intercessão entre os estudos ambientais e os sistemas complexos. Em 2021 a Academia Real das Ciências da Suécia concedeu metade do Prêmio Nobel de Física para Syukuro Manabe e Klaus Hasselmann, cujos estudos apresentam modelos complexos para a análise do clima. Em particular apontam uma correlação entre as emissões de dióxido de carbono e as mudanças climáticas.

A aquisição, manipulação, gestão, armazenamento e criação de valor a partir de dados, através de ambientes computacionais, tem-se apresentado como um novo paradigma tecnológico. Um campo do conhecimento que recebeu a denominação de Ciência de Dados, conceito que envelopa alguns termos frequentemente associados à inovação científica, técnica e social como \emph{Big Data}, mineração de dados, \emph{Business Intelligence} internet das coisas, inteligência artificial e aprendizado de máquina(AM), dentre outros \cite[p. 12-13]{EMCdata2015}.

As séries temporais são definidas como um conjunto de observações (numéricas ou categóricas) ordenado no tempo.  Embora muitos dos dados que descrevem as dinâmicas espaciais podem ser registrados na forma de sérias temporais (abastecimento de água nas tubulações, consumo de energia elétrica nos imóveis, fluxos de pessoas e veículos pela cidade, casos de uma doença por dia, etc.), contudo as técnicas de medição de correlações, bem como a devida exploração destas para inferir novos conhecimentos, permanecem como perguntas abertas em muitas sub-áreas das ciências ambientais\cite{Bermudez-Edo2018}.


\section{Definição do problema}
\label{sec:problema}

Os fenômenos climáticos apresentam as características dos sistemas complexos. Um sistema integrado, envolvendo aspectos globais e as condicionantes planetárias, fatores locais de cobertura da terra, proximidade de corpos d'agua, regime de ventos, dentre outros. Em alguns casos, podendo citar o bioma dominante de um determinado lugar, o fator tanto influencia o clima quanto é influenciado por ele.

As inter-relações entre as diversas variáveis climáticas não podem ser facilmente correlacionadas em grande escala. É possível estabelecer relações entre certas medidas meteorológicas em uma determinada localidade, ainda que as relações entre essas não sejam necessariamente relações que podem ser transportadas para toda e qualquer localidade do planeta. Mas a possibilidade de estabelecer relações em rede entre as variáveis climáticas de diferentes localidades umas com as outras ainda é um problema aberto.

\section{Importância da Pesquisa}
\label{sec:justificativa}



\section{Limites e Limitações}
\label{sec:limites}


\section{Questões e Hipóteses}
\label{sec:questoes}

É possível estabelecer e medir correlações entre aspectos climáticos de uma determinada localidade e um conjunto de outras localidades?

A hipótese é de que os fenômenos climáticos estão relacionados e forma complexa. Massas de ar percorrem distâncias na atmosfera, 

\section{Aspectos Metodológicos}
\label{sec:metodologia}


\section{Organização da Tese}
\label{sec:organizacao}
