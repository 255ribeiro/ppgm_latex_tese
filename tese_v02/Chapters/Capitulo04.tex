\chapter{Artigo 02: A Python/Zig optimized and customizable
implementation for \pdcca~and \dmc~coefficients}
\label{cap:paper_02}

\begin{flushright}
    ``Just remember, \\
    once you're over the hill\\
    you begin to pick up speed.''\\[10px]
    (Charles M. Schulz)
    \end{flushright}


O artigo \emph{A Python/Zig optimized and customizable
implementation for \pdcca~and \dmc~coefficients} está sendo revisado com o objetivo de ser publicado na revista \emph{Journal of Statistical Software}. Apresenta um pacote \emph{Python} com métodos de cálculo das funções \dfa~e \dcca e dos coeficientes \pdcca~e \dmc. Os cálculos mais custosos foram implementados em programação de nível mais baixo, utilizando uma linguagem de sistemas, nova e promissora, chamada \emph{Zig}~\url{https://ziglang.org/}. A escolha da linguagem \emph{Zig}~se deu por vários motivos: 

\begin{itemize}
\item Por ser uma linguagem de baixo nível, obtendo velocidades de processamento compatíveis com \emph{C} e \emph{Fortran}~\cite{10820804, Kacs_2024}.
\item Possui compatibilidade com \emph{C} e \emph{C++}, podendo incorporar bibliotecas.
\item Possui uma sintaxe moderna e elegante para tratar problemas de baixo nível.
\item Tem um eficiente sistema de \emph{cross-compilation}.
\end{itemize}

O aspecto das \emph{cross-compilation} foi um dos mais considerados na escolha da linguagem. A manutenção de uma biblioteca por um pequeno grupo de pessoas possui diversos desafios. A capacidade de fornecer programas compilados para diversos sistemas operacionais e arquiteturas de processadores é uma etapa crucial para a popularização de um pacote. A possibilidade de se gerar os executáveis desta biblioteca em um único computador pessoal pareceu uma opção vantajosa.

\begin{algorithm} \caption{Detrended Saved} \label{alg:det_reused}
  \begin{enumerate}[label=4.\alph*]
      \item \textbf{Calculando o \emph{Detrended Value}}: Para cada série $X^{j}$ (incluindo a variável dependente) onde  $1 \ge j \ge m$, sendo $m$ o número de séries temporais; para cada escala temporal, em cada caixa $i$, calcula-se o valor de $DV^{j}_{k,i} = (X^{j}_{k,i}-\widetilde{X^{j}}_{k, i})$ e armazena-se em uma matriz. A matrix tem por dimensões $m, n+1$, onde cada linha corresponde a uma série temporal e as colunas correspondem ao número de pontos em cada caixa.
      \item \textbf{Cálculo da função $f_{DFA}^{2}$ para cada caixa}: Calcula-se o valor do \dfa~:\\[10pt]
          $f_{DFA}^{2}(n, i) = \frac{1}{1+n} \sum_{k=i}^{i + n}(DV^{j1}_{k,i})^{2}$;
      \item \textbf{Cálculo da função $f_{DCCA}^{2}$~em cada caixa}: com a matriz $DV$ devidamente preenchida, para uma das $N - n$ caixas de uma mesma escala temporal a função é calculada para todas as combinações de séries temporais duas à duas por:\\[10pt]
          $f_{DCCA}^{2}(n, i) = \frac{1}{1+n} \sum_{k=i}^{i + n}(DV^{j1}_{k,i}) \times (DV^{j2}_{k,i})$
      \item \textbf{Retomando o algorítimo padrão}: Após o cálculo de todas as caixas, aplica-se o passo 5 do \dfa~e \dcca. 
  \end{enumerate}
  \end{algorithm}

O \emph{software} de cálculo de dinâmica de fluidos computacional \emph{AeroSim}~\url{https://aerosim.io/} apresenta desempenho e precisão nos seus cálculos \cite{romanusViableFrameworkWind2023, lugariniLargeEddySimulations2024} é parcialmente implementado em \emph{Zig} e serviu de incentivo à esta adoção.

Parte da otimização segue a implementação de \citeonline{hartmannRealtimeFractalSignal2013} para o \dfa, transposta para o \dcca~e \pdcca~por~\citeonline{Kapostza2022}. Onde, durante o cálculo da interpolação da reta de tendência, na primeira caixa de cada séries, pelo método dos mínimos quadrados, valores são gravados em variáveis que podem tornar o \emph{loop} de somatórios dos valores utilizados no cálculo dos coeficientes da equação da reta são armazenados em variáveis. Para a caixa seguinte, o \emph{loop} é substituído por operações de subtração e adição.



A Equação~\ref{eq:sum_opt_x}, mostra que o valor da soma das coordenadas no eixo das abscissas de uma caixa pode ser substituído pelo somatório total na caixa anterior menos o primeiro valor de abscissa da caixa anterior, somado ao último valor de $x$ para a caixa atual. Raciocínio análogo pode ser aplicado às Equações~\ref{eq:sum_opt_x2}, \ref{eq:sum_opt_y}~e~\ref{eq:sum_opt_xy}.

\begin{equation}
    \label{eq:sum_opt_x}
    \forall~1<i\leq(N-n),~\sum_{k=i}^{i + n}T_k = \left(\sum_{j=i-1}^{(i+n)-1}T_j\right)~-~T_{i-1}~+~T_{i + n}
  \end{equation}
  
  \begin{equation}
    \label{eq:sum_opt_x2}
    \forall~1<i\leq(N-n),~\sum_{k=i}^{i + n}T_k^2 = \left(\sum_{j=i-1}^{(i+n)-1}T_j^2\right)~-~T_{i-1}^2~+~T_{i + n}^2
  \end{equation}
  
  \begin{equation}
    \label{eq:sum_opt_y}
    \forall~1<i\leq(N-n),~\sum_{k=i}^{i + n}S_k = \left(\sum_{j=i-1}^{(i+n)-1}S_j\right)~-~S_{i-1}~+~S_{i + n}
  \end{equation}
  
  \begin{equation}
    \label{eq:sum_opt_xy}
    \forall~1<i\leq(N-n),~\sum_{k=i}^{i + n} (S_k\times T_k) = \left(\sum_{j=i-1}^{(i+n)-1}(S_j \times T_j)\right)-(S_{i-1} \times T_{i-1})+(S_{i + n} \times T_{i + n})
  \end{equation}

Quando se pensa no cálculo do \dmc, e em como otimiza-lo, deve-se pensar em múltiplos cálculos de \pdcca, para a criação da matriz apresentada na Equação~\ref{eq:p_dcca_matrix}. A ideia é substituir o passo 4 do Algoritmo~\ref{alg:dcca} pelos passos descritos no Algoritmo~\ref{alg:det_reused}.



Essa pequena mudança, embora possa até aumentar o tempo de processamento para um pequeno número de séries temporais, demonstra-se muito vantajoso quanto maior for o número de séries cujo coeficiente \pdcca~ precisa ser calculado.

\begin{equation}\label{eq:combinations_2x2}
  \frac{j!}{2 \times (j-2)!}
\end{equation}

Na biblioteca \emph{Zebende} é possível implementar outras versões do código para o cálculo do \pdcca~com poucas séries temporais. Mas é preciso entender em que ponto o Algoritmo~\ref{alg:det_reuse} passa a ser vantajoso. Caso o $DV$ não seja armazenado, o número de vezes que ele tem que ser calculado é de duas vezes o número de combinações duas a duas para $j$ séries~(Equação~\ref{eq:combinations_2x2}) e o tempo gasto para calcular o $DV$ depende do tamanho das séries.

Vale também ressaltar que as otimizações baseadas nos parâmetros do método dos mínimos quadrados pode ser adaptado para diversos graus para o polinômio da tendência, mas não pode se adaptar à caixas não sobrepostas.

Como apresentado na Sessão\ref{ss:dfa_fract} do Capítulo~\ref{cap:fund_teorica}, alguns autores ainda advogam pela não sobreosição das caixas~\cite{zhouMultifractalDetrendedCrosscorrelation2008}, nestes casos, o armazenamento dos parâmetros para o cálculo dos mínimos quadrados não funcionaria. Por outro lado, para uma grande quantidade de séries temporais o Algorítimo~\ref{alg:det_reused}, \emph{Detrended Saved}, funcionaria também no cenário de caixas não sobrepostas.

Além da otimização, o artigo apresenta o modo de uso da biblioteca e algumas das funções auxiliares já implementadas. As equações destacadas nas Sessões~\ref{ss:dfa_fract}~e~\ref{ss:vari_cross} do Capítulo~\ref{cap:fund_teorica} estão entre as próximas adições ao pacote. 


    \includepdf[pages=-, pagecommand={\thispagestyle{plain}}]{./Papers_publi/paper_zebendelib.pdf}