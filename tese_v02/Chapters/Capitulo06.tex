\chapter{Conclusão}
\label{cap:coclusion}

\begin{flushright}
    ``I live on Earth at present, and I don’t know what I am. \\
    I know that I am not a category. I am not a thing -- a noun.\\ 
    I seem to be a verb, an evolutionary process (...)''\\[10px]
    (R. Buckminster Fuller)
    \end{flushright}

A pesquisa apresenta resultados sólidos e formulações de novas perguntas. Valores de funções e coeficientes foram calculados e testados. Resultados de uma pesquisa, realizada com dados de EEG foram validados pelos pares em publicação.

Uma ferramenta computacional para proporcionar facilidade e velocidade nos cálculos do \dmc~foi implementada e trouxe consigo um novo algoritmo, que apresenta vantagens de desempenho em certos casos de uso.

A mudança na quantidade de informação que pode ser tratada,utilizando a ferramenta, foi ilustrada e critérios para definição de semelhança entre séries temporais foram estabelecidos e avaliados.

Entende-se que, considerando o exposto na Sessão~\ref{ss:aplica} do Capítulo~\ref{cap:fund_teorica}, pela grande aplicabilidade do método em estudos da área de Ciências da Terra e do Ambiente, os avanços aqui apresentados contribuem com a área.

\section{Trabalhos futuros}

O pacote \emph{Zebende} continua sua evolução, mudanças na usabilidade de certas funções serão implementadas, assim como novos algoritmos devem fazer parte da biblioteca. Tendo os algoritmos destacados nas Seções~\ref{ss:dfa_fract} e \ref{ss:vari_cross} como os primeiros a serem incorporados à ferramenta.

A possibilidade de, através da ferramenta computacional criada, explorar critérios de proximidade entre séries temporais e a sua aplicabilidade como seleção de atributos para algoritmos de aprendizado de máquina deve ser melhor explorada.