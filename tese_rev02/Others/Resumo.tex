\begin{thesisresumo}

Esta Tese apresenta uma investigação sobre análise de séries temporais através das funções \dfa, \dcca~e, principalmente nos coeficientes \pdcca~e \dmc. Apresenta uma pesquisa bibliográfica focada em métodos baseados no \dfa~que tratam de multi-correlação e as aplicações destas funções e coeficientes nas Ciências Ambientais. Apresentamos também artigos, produzidos no processo desta tese, tratando da aplicação dos coeficientes \pdcca~e \dmc.
    
A implementação de uma ferramenta computacional (a biblioteca \emph{Python} \emph{Zebende}) para a manipulação das séries temporais e cálculo das funções e coeficientes é apresentada , assim como o algorítmo \emph{Detrended Saved}, uma estratégia inovadora no cálculo do \dcca de forma geral, mas com vantagens no desempenho quando utilizado para cálculos de muitas séries entre si (como no caso da montágem da matriz do \pdcca~para calcular o \dmc, quando um grande número de séries temporais é utilizado). A capacidade da biblioteca \emph{Zebende} potencializar a utilização destes coeficientes também é apresentada.
\\

\textbf{Palavras Chaves:} \ppgmpalavraschave

\end{thesisresumo}
